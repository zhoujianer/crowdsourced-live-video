%!TEX root = main.tex

\section{Findings and Implications}
\label{sec:summary}

\begin{table}
\tablefontsize
\renewcommand{\arraystretch}{\assize}
\setlength{\tabcolsep}{3pt}
\caption{Summary of findings and their implications.}
\centering
\begin{tabular}{L{3.2in}|L{3.2in}}
	\toprule
	\multicolumn{1}{c|}{Findings} & \multicolumn{1}{c}{Implications} \\
	\hline
	37\% of stalls in upload flows cause the download flows to have no data to transmit and thus result in a stall in download flows. & Optimizing of upload network and the performance of upload flows may greatly improve the performance of download flows. \\
	\hline
	When reordering rate in upload flow grows, the FCT of upload flows decreases greatly, which means the users tend to terminate video sharing if the network quality becomes bad. & Service provider could deploy front-end servers more nearer to video sharing users to optimize the upload performance, which may greatly improve the Quality of Experience (QoE) for both upload and download users. \\
	\hline
	Timeout retransmission contributes 29.8\% of stalls in download process. & Costly timeout retransmissions in short flows could be eliminated through slightly aggressively retransmission strategies, \eg \cite{flach2013reducing,zhou2015demystifying}. \\
	\hline
	Different types of frame experience different timeout retransmission characteristics due to their distinct frame production speeds and frame sizes, \eg I frame experiences more \emph{continuous loss} stalls while V frame experiences more \emph{less pkt retrans} stalls. & Transmit each type of frame in separate connections, and optimize each connection by eliminating the timeout retransmissions according to the characteristics of frames and connections. \\
	\bottomrule
\end{tabular}
\label{tbl:summary}
\termspace
\end{table}

Table~\ref{tbl:summary} summarizes the main findings and implications of our measurement results. Overall, our major findings could be grouped into two categories, one is the significant impact of upload performance on the involvement of view sharing and on the Quality of Experience of video viewing, the other is the distinct distribution of timeout retransmission stalls in different types of frame.

As the upload performance is vital to the overall performance of the crowdsourced live video system, more front-end servers should be deployed in different locations and the nearest one with better network quality (\ie lower delay and lower packet loss rate) should be selected for the video sharing users, to gain a better upload performance.

Even through previous work \cite{flach2013reducing,zhou2015demystifying} found the great degradation of performance of short flows due to timeout retransmission, we in this paper further find that flows may exhibit distinct characteristic of timeout retransmission due to different frame production intervals and frame sizes. To eliminate these costly timeout retransmissions, one could adopt existing solutions like TLP~\cite{flach2013reducing}, S-RTO~\cite{zhou2015demystifying}. Service provider could also optimize the performance by transmitting different frames in separate connections. Since the drop of a P or V frame can only hurt the watching experience slightly, P and V frames could be delivered in UDP connections and I frame is transmitted in TCP connection. Divide-and-conquer optimization of each type of connection may bring more performance improvement of the overall system.