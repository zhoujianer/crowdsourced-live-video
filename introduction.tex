%!TEX root = main.tex

\section{Introduction}
\label{sec:introduction}

 As cell phones and intelligent medium devices have camera, people can use such devices to broadcast their own activities to the Internet and we call such video as crowdsourced live video. Now more and more people have involved in the such kind of video as the emergence of popular application, such as Google Glass, Twitch.tv, Intelligent monitor~\cite{ishimaru2014blink}. People make use of crowdsourced live video to monitor their houses, make social activities with other people and even advertise their business. 
 
 Traditional live video system can be divided into two types. One has stable providers and multiple viewers, such as YouTube and game broadcaster~\cite{Mukerjee2014CDNvideo} ~\cite{yin2009livesky}. In these live video systems, once the video is produced, it is delivered to CDN(Content Delivery Network) and viewers fetch the videos from the CDN. The other type has unstable provider and single viewer, such as FaceTime and Skype~\cite{Xu2012skype}. For viewer, such video is unstable as it can not be delivered to CDN first and has to go through WAN(Wide Area Network) entirely the path~\cite{Benson2010DCN}. However there are just single viewer and the video provider can deliver the video directly to the viewer. Crowdsourced live video is different from traditional live video as it has both unstable providers and multiple viewers.
 
 In crowdsourced live video system, anyone may post their video. System operators can not be sure which video will be popular and how long the video is, so it is unreasonable to deliver  the video to CDN. Viewers have to fetch the video from where it is produced which means the video is unstable for viewers. Multiple viewers mean that the providers can not make connections directly with viewers as the providers are often low power and low processing ability handheld devices. Such different features make the traditional system hard to meet the crowdsourced live video's needs. It is necessary to detail the architecture of such crowdsourced live systems and figure out how well they work.

There has been some work focused on the effectiveness and measurement for the crowdsourced video system. Simoens \etal~\cite{simoens2013sigasight} propose a system to storage and search the crowdsourced videos. However, the system does not focus on video broadcast process. Zhang \etal~\cite{zhang2015twitch} collect the traffic on client side to 'guess' the architecture of the crowdsourced video system and measure the user behaviour. Until now, there are not detail introduction to the crowdsourced live video system and the performance of such system has not been analysis in the wild. Motivated by this, in this paper we introduce the architecture of a crowdsourced live video system which has been deployed in the production environment and measure the effectiveness of this system by large packet-level traces collected in the front-end relay servers. In particular, our main observations are the following:

\newcommand{\squishlist}{
 \begin{list}{$\bullet$}
  { \setlength{\itemsep}{0pt}
     \setlength{\parsep}{3pt}
     \setlength{\topsep}{3pt}
     \setlength{\partopsep}{0pt}
     \setlength{\leftmargin}{1.5em}
     \setlength{\labelwidth}{1em}
     \setlength{\labelsep}{0.5em} } }

\newcommand{\squishend}{
  \end{list}  }

\squishlist
\item By using a front-end relay server, the crowdsourced live video can effectively upload and by caching the video temporarily on the front-end server the system can handle mulitple viewers scalable.

\item As the upload flows'  reorder rate grows, the flow rate drops greatly and viewers trend to stop uploading their video.

\item About 40\% of stalls on the upload process will cause the download flows do not have data to send and result in a stall in download flows.

\item In video flows, I frame(the most important frame in H.264) has the highest packet loss rate compared with P and V frames.
    
\squishend 

The rest of the paper is structured as follows. Section~\ref{sec:system} introduces the crowdsourced live video system and the overview of video frame. Section~\ref{sec:dataset} presents how we collect the dataset and the basic information of the datasets. In Section~\ref{sec:system-measurement} we measure the system's upload, download and their relationship in detail. Section~\ref{sec:frame-analysis} analysis the feature of different frames in H.264. Section~\ref{sec:summary} summarize the discovery we find and Section~\ref{sec:conclusion} concludes the paper. 